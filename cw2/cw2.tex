\documentclass[12pt, letterpaper, titlepage]{article}
\usepackage[left=3cm, right=2cm, top=2cm, bottom=2cm]{geometry}
\usepackage[MeX]{polski}
\usepackage[utf8]{inputenc}
\usepackage{graphicx}
\usepackage{enumerate}
\usepackage{amsmath} %pakiet matematyczny
\usepackage{amssymb} %pakiet dodatkowych symboli
\title{Sport - zdrowie i nie tylko}
\author{Michał Plona}
\date{15.10.2022}
\begin{document}
\maketitle
\begin{center}
Spis Treści
\end{center}
\begin{enumerate}[1]
\item Jakie korzyści może nam przynieść sport?
\item Sport - dla każdego coś odpowiedniego
\begin{enumerate}[2.1]
\item Jazda na rowerze
\item Bieganie
\item Joga
\item Wspinaczka skałkowa
\item Taniec
\end{enumerate}
\item Jeśli to dla Ciebie ważne, znajdziesz sposób, jeśli nie, znajdziesz wymówkę!
\end{enumerate}
\newpage
\begin{center}
\section*{Jakie korzyści może nam przynieść sport}
\end{center}
Jedną z podstawowych korzyści, jaką daje nam sport, jest mniejsze prawdopodobieństwo zapadnięcia na wiele chorób. Aktywność fizyczna:
\begin{enumerate}[-]
\item powoduje obniżenie poziomu cukru we krwi
\item reguluje przemianę materii
\item obniża ryzyko powstawania zaburzeń sercowo-naczyniowych
\item obniża ciśnienie krwi
\item zapobiega osteoporozie
\item wzmacnia system immunologiczny
\end{enumerate}
Dodatkową korzyścią z podejmowania aktywności fizycznej jest utrzymanie prawidłowej wagi ciała. Podczas ćwiczeń nasila się szybkość spalania kalorii, co wpływa na redukcję tłuszczu i trwałe osiągnięcie idealnej sylwetki.
\newline
\newline  Kolejną istotną korzyścią wynikającą z uprawiania sportu jest poprawa nastroju, ponieważ:
\begin{enumerate}[-]
\item poprzez ruch polepsza się nasza wydajność umysłowa, czyli zwiększa się zdolność zapamiętywania
\item regularne ćwiczenia obniżają prawdopodobieństwo zachorowania na depresję i inne zaburzenia nastroju - dzieje się tak, ponieważ podczas ćwiczeń wydziela się serotonina odpowiedzialna za poczucie szczęścia
\item sport daje poczucie satysfakcji, przez co wzrasta nasza samoocena
\item poprzez ruch rozładowujemy negatywne emocje
\end{enumerate}
\newpage
\begin{center}
\section*{Sport - dla każdego coś odpowiedniego}
\end{center}
Aby czerpać przyjemność z ćwiczeń, ważne jest odpowiednie dobranie zajęć sportowych - nie tylko do naszych fizycznych możliwości, ale także do preferencji i osobowości. Dlatego też dla osób spokojnych i precyzyjnych lepsze będą zajęcia jogi, zaś dla osób odczuwających nadmiar energii milsze będą aerobik, siłownia lub bieganie.
\newline
\subsection*{Jazda na rowerze}
Jazda rowerem może być czystą rekreacją, a przy tym sposobem na dotarcie do celu. Wiele osób woli przesiąść się z samochodu na dwa kółka, gdyż oprócz uniknięcia korków jazda na rowerze daje wiele korzyści ogólnozdrowotnych:
\begin{enumerate}[-]
\item chroni przed miażdżycą
\item powoduje wzrost wytrzymałości mięśni
\item obniża poziom złego cholesterolu i podwyższa poziom dobrego
\item zwiększa pojemność płuc, sprawia, że krew jest bardziej bogata w tlen, a serce pracuje dużo lepiej
\item polepsza samopoczucie, zmniejsza napięcie i długotrwały stres
\end{enumerate}
\subsection*{Bieganie}
Bieganie, jak każda aktywność fizyczna, poprawia naszą kondycję, zdrowie i samopoczucie. Na dodatek to aktywność, która nie wymaga od nas drogiego sprzętu i można uprawiać ją dosłownie wszędzie.
\newline
Regularne bieganie:
\begin{enumerate}[-]
\item dotlenia mózg
\item podnosi wytrzymałość
\item przeciwdziała depresji
\item chroni przed chorobą wieńcową
\item spala kalorie - w przypadku treningów interwałowych nawet przez długi czas po wysiłku
\end{enumerate}
\subsection*{Joga}
Joga to ćwiczenia dla każdego, niezależnie od wieku czy sprawności fizycznej. Poziom trudności można dostosować do indywidualnych możliwości danej osoby. Joga jest zespołem ćwiczeń fizyczno-umysłowych, mającym na celu relaks, rozluźnienie umysłu i ciała. Poprzez praktykowanie jogi kształtujemy postawę, a stawy zyskują większą sprężystość.
\newline
Oprócz tego joga:
\begin{enumerate}[-]
\item łagodzi stres
\item poprawia krążenie
\item uspokaja
\end{enumerate}
\subsection*{Wspinaczka skałkowa}
Choć ten rodzaj sportu jest jednym z bardziej wymagających, a do jego uprawiania potrzebujemy silnych mięśni rąk oraz dobrej kondycji fizycznej, to cieszy się rosnącą popularnością. Wspinaczka pozwala spalać kalorie, odstresować się, a przy tym dostarcza wielu wrażeń. Innymi ważnymi zaletami jest to, że wspinaczka:
\begin{enumerate}[-]
\item poprawia elastyczność ciała
\item rozwija mięśnie
\item poprawia koordynację
\end{enumerate}
\subsection*{Taniec}
Modną formą rekreacji ruchowej dla osób w każdym wieku jest taniec. Oferuje on aktywność fizyczną tym, którzy nie czują się sportowcami. Taniec pozwala uzyskać i utrzymać formę, poprawia elastyczność ciała oraz ułatwia poznanie własnego ciała, poprawiając postawę i równowagę. Taka forma aktywności umożliwia również poznawanie nowych ludzi, którzy posiadają podobne zainteresowania - w ten sposób taniec pomaga budować umiejętności społeczne. Poprzez taniec zwiększamy także pewność siebie.
\newpage 
\begin{center}
\section*{Jeśli to dla Ciebie ważne, znajdziesz sposób, jeśli nie, znajdziesz wymówkę!}
\end{center}
Pomimo tak wielu korzyści płynących z uprawiania sportu wiele osób wybiera bierny tryb życia. Często można usłyszeć, że ktoś chciałby zacząć ćwiczyć, ale sport nie jest dla niego, ponieważ brak mu na to czasu. Jedną z najczęstszych wymówek jest to, że po powrocie do domu, po ciężkim dniu w pracy czeka na nas masa innych rzeczy do zrobienia, a potem jesteśmy już tak zmęczeni, że nie mamy siły na jakiekolwiek ćwiczenia. Tak naprawdę jednak między tymi ważnymi sprawami poświęcamy mnóstwo czasu na te mniej ważne - siedzenie przed telewizorem czy komputerem. A możliwe jest takie dobranie ćwiczeń, aby zrelaksować się i móc czerpać z nich przyjemność.
\newline
\newline
\textbf{Sport} daje nam więc wiele korzyści zdrowotnych, polepsza nasz nastrój, pomaga utrzymać szczupłą sylwetkę, a nawet poprawia naszą zdolność komunikowania się w grupie.  Każdy z nas jest inny, każdy ma inną osobowość, inne zainteresowania i upodobania. Mimo to w kwestii sportu każdy może znaleźć coś dla siebie, ponieważ dziedzin i dyscyplin istnieje bardzo wiele. Naprawdę warto, a więc do dzieła!
\end{document}

